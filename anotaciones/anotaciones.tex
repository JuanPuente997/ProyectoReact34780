
npx create-react-app nombredelproyecto

si creo un componente en un archivo js 
export default nombreDelArchivojs

en App.js
import nombreDelArchivojs from './nombreDeLaCarpeta'

para agregar estilos a ese componente creado con js
importo al archivo js el link del archivo css
import'./nombreDelArchivo.css'

Datos prmitivos
Hay 6 tipos de datos primitivos: string, number, bigint, boolean, undefined y symbol.



                operador ternario
const condition = true
let result
result = condition? 'correcto' : 'incorrecto'
console.log('el resultado es: ' + result 


                spread operator
...array
const nums = [1,2,3,4]
const nums2 =[...nums,5] 

nums2 obtiene los mismos valores que el primer array pero
es modificado sin afectar al primer array

const persona = {
    name:'carlos',
    age: 30,
    prhone:'1234'
}

const persona2 = {
...persona,
name:'pepe',     <- piso el valor y mantengo el resto
}
-----------------------------------------------------
const persona2 = {
    age: persona.age <- paso solo el valor de la edad
}
------------------------------------------------------
const {name} = persona <- extraigo un solo valor de la constante 
------------------------------------------------------
const {name:name2} = persona2 <- desestructuro el valor de una propiedad con el mismo nombre que el de otra constante asignandole otro nombre
-------------------------------------------------------
                        "eliminar" propiedad
const empleado = {
    name: 'carlos',
    age: 30,
    position: 'albañil'
}
const {position: propiedadEliminada, ...restoDePropiedades}= empleado

console.log(restoDePropiedades) <- se va a consologuear el nombre y la edad y la position se almacena en una "variable" que nunca se muestra. esto tambien seria desestruturacion 
-----------------------------------------------------------
                        eliminar de forma dinamica
const key = 'position' <- selecciono de forma dinamica la propiedad a eliminar

const{[key]:propiedadEliminada, ...restoDePropiedades}= empleado
console.log(propiedadEliminada) <- la position se 'elimina'
console.log(restoDePropiedades) <- aca se guarda el resto de valores

 

            (retrocompatibilidad) Polyfills
un bloque de codigo que le agrega funcionalidad a mi entorno de ejecucion 

const num = 9
const num2 = 29

number.prototype.multiplicar = function (numero){
    let acumulador = 0
    for (let i = 0; i < numero; i++){
        acumulador += this         <-  (hace referencia a num y num2)
    }
    return acumulador
}

const result = num.multiplicar(8)
const result2 = num2.multiplicar(10)
console.log(result)
console.log(result2)


                    JSX
<div className= "active"> sdad</div>

React.createElement ('div,{className:'active'}'sdad')
 

                ESTILOS EN JSX
let styles = {
    borderColor: '#999'
}
const jsx = (
    <div syle={styles}>
    sdad
    <div>
)
-----------------------------------------------